\documentclass[a4paper]{article}
\usepackage{fontspec}
\usepackage{listings}
\usepackage{graphicx}
\usepackage{fancyhdr}
\usepackage{setspace}
\usepackage{float}
\usepackage{colortbl}
\usepackage{color,xcolor}
\usepackage{longtable}
\usepackage{array,booktabs}
\def\oo{\nolinebreak[4]\hspace{.3em}\raise.7ex\hbox{。}\hspace{-0.3em}}
\def\pp{\nolinebreak[4]\hspace{.3em}\raise.6ex\hbox{,}\hspace{-0.3em}}
\def\dd{\nolinebreak[4]\hspace{.3em}\raise.8ex\hbox{、}\hspace{-0.1em}}
\def\mm{\nolinebreak[4]\hspace{.3em}\raise.3ex\hbox{;}\hspace{0.3em}}
\def\kk{\nolinebreak[4]\hspace{.3em}\raise.3ex\hbox{:}\hspace{-0.3em}}
\setmainfont{SimSun}
\usepackage[margin=2cm]{geometry}

\definecolor{codegreen}{rgb}{0,0.6,0}
\definecolor{codegray}{rgb}{0.5,0.5,0.5}
\definecolor{codepurple}{rgb}{0.58,0,0.82}
\definecolor{backcolour}{rgb}{0.95,0.95,0.92}

\lstdefinestyle{mystyle}{
    backgroundcolor=\color{backcolour},   
    commentstyle=\color{codegreen},
    keywordstyle=\color{magenta},
    numberstyle=\tiny\color{codegray},
    stringstyle=\color{codepurple},
    basicstyle=\ttfamily\footnotesize,
    breakatwhitespace=false,         
    breaklines=true,                 
    captionpos=b,                    
    keepspaces=true,                 
    numbers=left,                    
    numbersep=5pt,                  
    showspaces=false,                
    showstringspaces=false,
    showtabs=false,                  
    tabsize=4
}


\begin{document}

    \newpage
    \thispagestyle{empty}
    \large
    \vspace{20cm}\begin{center}
        \hspace{-1em}投稿類別\kk 資訊類\\[3cm]
        篇名\kk \\
        加密貨幣量化交易\\[5cm]
        作者\kk \\
        柯盛乾\oo 高雄師範大學附屬高級中學\oo 高二信班\\[3cm]
        指導老師\kk \\
        老師
    \end{center}



    \newpage
    \thispagestyle{plain}
    \small\hspace{17em}加密貨幣量化交易
    \setcounter{page}{1}
    \large

    壹\dd 前言\\  

    \hspace{-1.2em}一\dd 研究動機\\
    

    \hspace{0.8em}近幾年加密貨幣的市值快速上升\pp 加密貨幣市場的浮動很大\pp 操作的機會也\\
    \hspace{1em}相對較多\pp 但是隨浮動大而來的還有風險\oo 一個浮動大的市場常會讓經驗不足的\\
    交易者無法冷靜思考市場的方向\pp 或是做出不理性的交易策略\oo 除此之外\pp 因為\\
    市場每天24小時都在運作\pp 在睡覺時可能錯失良好的交易機會\oo 再加上平時要上\\
    課\pp 無法時時盯盤\oo 種種原因引發我對程式交易的研究\pp 希望藉此做出一套全自\\
    動且可盈利的交易機器人\\[0.5cm]

    \hspace{-1.2em}二\dd 研究目的\\

    \hspace{0.8em}應用所學\pp 跨領域結合\pp 我希望結合程式與金融\pp 在過程中邊做邊學\pp 也累\\
    積一些實作成品\pp 強化自己的興趣\pp 進而將它當作專業之一\\

    \hspace{-1.2em}三\dd 研究步驟\\

    \vspace{-2em}\begin{figure}[H]
        \centering
        \graphicspath{pic.jpg}
        \includegraphics[scale=0.35]{pic.jpg}
    \end{figure}

    \vspace{-1.5em}貳\dd 正文\\

    一\dd 研究設備\\

    \vspace{-1em}\hspace{10.5em}表一 設備\dd 軟體明細表\\

    \vspace{-0.5em}\hspace{6em}\begin{tabular}{cc}
        \toprule  
        設備名稱& 規格\\
        \midrule 
        中央處理器 CPU &  intel core i5\\
        記憶體 RAM &  16GB\\
        作業系統 & window10\\
        程式語言 & python3.7(64-bit)\\
        編輯環境 & visual studio code 1.64\\
        看盤軟體 & TradingView1.22\\
        \bottomrule 
    \end{tabular}\\

    二\dd 指標\dd 策略介紹\\

    (一)ATR指標算法介紹\\
    ATR指標(Average True Range)是構成supertrend指標的重要參數\pp 在了解supertrend的\\
    算法前\pp 必須要知道ATR的算法\pp 在計算ATR指標時\pp 要先算出TR(True Range)這個數列\\
    ATR和TR是由如下方式計算\kk 
    \[
        TR=MAX[(High-Low),abs(high-Close_{prev}),abs(Low-Close_{prev})]
    \]
    \[
        AT\!R=(\frac{1}{n}) \sum_{i = 1}^{n}  (T\!R_i)  
    \]
    \[
        (n表示AT\!R的週期\pp T\!R_i為該日的TR值)
    \]


    \newpage
    \thispagestyle{plain}
    \small\hspace{17em}加密貨幣量化交易
    \chead{加密貨幣量化交易}
    \setcounter{page}{2}
    \large\\

    
    (二)supertrend指標算法介紹\\
    supertrend是由ATR和一個常數multiplier組成\pp 計算方式如下\kk

    \[
        Up = (high + low) / 2 + multiplier  \times  ATR
    \]
    \[
        Down = (high + low) / 2 – multiplier \times ATR
    \]
    當價格(收盤價)高於上軌時\pp 趨勢反轉\pp 下軌出現\oo 當價格(收盤價)低於下軌時\pp 
    趨勢反轉\pp 上軌出現\pp 周而復始\oo 如下圖
    
    \vspace{2em}\begin{figure}[H]
        \centering
        \graphicspath{TV2.png}
        \includegraphics[scale=0.9]{TV2.png}
    \end{figure}
    \hspace{8em}圖(一)supertrend在看盤軟體可視化

(三)Heikin Ashi candle(平均k線)介紹\\
    因為平均K線的算法不同\pp 看起來會更為平滑\pp 以下是平均K線的算法\\
    和其可視化的表述
    \[
        HeikinAshi_{close}=\frac{Open_0+Close_0+High_0+Low_0}{4}
    \]\\[-2em]
    \[
        HeikinAshi_{open}=\frac{H\!A_{Open_{-1}}+H\!A_{Close_{-1}}}{2}
    \]
    \[
        HeikinAshi_{high}=MAX(High_0,H\!A_{Open_0},H\!A_{Close_0})
    \]\\[-2em]
    \[
        HeikinAshi_{low}=min(Low_0,H\!A_{Open_0},H\!A_{Close_0})
    \]
    
    \vspace{2em}\begin{figure}[H]
        \centering
        \graphicspath{TV3.png}
        \includegraphics[scale=0.9]{TV3.png}
    \end{figure}
    \hspace{9em}圖(二)平均K線在看盤軟體可視化
    


\newpage
\thispagestyle{plain}
\small\hspace{17em}加密貨幣量化交易
\chead{加密貨幣量化交易}
\setcounter{page}{3}
\large\\




(四)策略雛型\\

純粹使用K線的數據帶入所算出的supertrend並不是特別理想\pp 因此使用平均K線\\
的開高低收當作supertrend的參數\pp 這樣所算出的supertrend也相對一致\oo\\
對震盪的敏感度不會太強烈\pp 這樣也會比較好跟住市場趨勢\oo 圖(三)\dd 圖(四)是相同參數\dd \\
相同時間\dd 相同交易對(AVAX/USDT)下\pp 一般K線與平均K線的對比案例

\vspace{1em}\begin{figure}[H]
    \graphicspath{TV7.png}
    \hspace{3em}\includegraphics[scale=0.65]{TV7.png}
    \hspace{1in}
    \hspace{-3em}\includegraphics[scale=0.63]{TV6.png}
\end{figure}
\hspace{2em}圖(三)一般K線搭配supertrend \space \space \space \space \space \space \space \space \space \space 圖(四)平均K線搭配supertrend
\\


三\dd 平均K線與supertrend的python程式碼
\vspace{1em}\hspace{10em}\lstset{style=mystyle}
\begin{lstlisting}[language=Python]
import pandas as pd
import pandas_ta as ta  
import numpy as np

def Supertrend(df,high,low,close,atr_period,multiplier):

    price_diffs = [ high-low, 
                    high-close.shift(), 
                    close.shift()-low]
    true_range=pd.concat(price_diffs, axis=1)
    true_range=true_range.abs().max(axis=1)
    atr=true_range.ewm(alpha=1/atr_period,min_periods=atr_period).mean() 
    hl2=(high+low)/2
    final_upperband=upperband=hl2+(multiplier*atr)
    final_lowerband=lowerband=hl2-(multiplier*atr)
    
    supertrend=[True]*len(df)
    
    for i in range(1,len(df.index)):
        curr,prev=i,i-1
        
        if close[curr]>final_upperband[prev]:
            supertrend[curr]=True
        elif close[curr]<final_lowerband[prev]:
            supertrend[curr]=False
        else:
            supertrend[curr]=supertrend[prev]
            if supertrend[curr]==True and final_lowerband[curr]<final_lowerband[prev]:
                final_lowerband[curr]=final_lowerband[prev]
            if supertrend[curr]==False and final_upperband[curr]>final_upperband[prev]:
                final_upperband[curr]=final_upperband[prev]

        if supertrend[curr]==True:
            final_upperband[curr]=np.nan
        else:
            final_lowerband[curr]=np.nan

    return df.join(pd.DataFrame({
        'Supertrend':supertrend,
        'Final Lowerband':final_lowerband,
        'Final Upperband':final_upperband
    },index=df.index))


def HA(df):
    ha_df=ta.ha(df["open"], df["high"], df["low"],df["close"])
    df=df.join(ha_df)
    return df

\end{lstlisting}
\newpage
\thispagestyle{plain}
\small\hspace{17em}加密貨幣量化交易
\chead{加密貨幣量化交易}
\setcounter{page}{3}
\large

    
    

\newpage
\thispagestyle{plain}
\small\hspace{17em}加密貨幣量化交易
\chead{加密貨幣量化交易}
\setcounter{page}{5}
\large\\

\vspace{-5em}\begin{figure}[H]
    \centering
    \graphicspath{paper1.png}
    \includegraphics[scale=0.7]{paper1.png}
\end{figure}
\end{document}

\[
    HeikinAshi_{close}=\frac{Open_0+Close_0+High_0+Low_0}{4}
\]\\[-2em]
\[
    HeikinAshi_{open}=\frac{H\!A\!-\!Open_{-1}+H\!A\!-\!Close_{-1}}{2}
\]
\[
    HeikinAshi_{high}=MAX(High_0,H\!A\!-\!Open_0,H\!A\!-\!Close_0)
\]\\[-2em]
\[
    HeikinAshi_{low}=min(Low_0,H\!A\!-\!Open_0,H\!A\!-\!Close_0)
\]
\[
    Up = (high + low) / 2 + multiplier  \times  ATR
\]
\[
    Down = (high + low) / 2 – multiplier \times ATR
\]
